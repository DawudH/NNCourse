\documentclass[journal,onecolumn]{IEEEtran}
%
% If IEEEtran.cls has not been installed into the LaTeX system files,
% manually specify the path to it like:
% \documentclass[journal]{../sty/IEEEtran}


% Some very useful LaTeX packages include:
% (uncomment the ones you want to load)

\usepackage{placeins}
\usepackage{color}
\usepackage{float}
\usepackage{booktabs}
\usepackage{pgfplots}
\usepackage{graphicx}
\usepackage{caption}
\usepackage{subcaption}
\usepackage{wrapfig}
\usepackage[hidelinks]{hyperref} % Clickable links in PDF.

\newlength\figureheight 
\newlength\figurewidth


% Set sizes figures (all the boxplots in 1 go!)
\setlength\figureheight{0.3\textwidth}
\setlength\figurewidth{0.4\textwidth}

\graphicspath{{./Figure/}}

% FOR CAPTION CENTERING NOT IN LINE WITH THE IEEETRAN STYLE!
%\usepackage{caption}% http://ctan.org/pkg/caption 


\usepackage[cmex10]{amsmath}



\begin{document}

\title{\LARGE Using a deep neural network for automatic colorization of greyscale images. {\color{red} Should describe what we exactly did... }}

\author{J.L. Dorscheidt (4091981), Dawud Hage (4190696), Dennis van der Hoff (4139925), Joost Meulenbeld (4103548)}% <-this % stops a space
% make the title area
\maketitle

% As a general rule, do not put math, special symbols or citations
% in the abstract or keywords.
%
\begin{abstract}
Automatic colorization of grayscale images with no further user input has several interesting applications, of which grayscale video may be the most useful one. In this paper a comparison is made between several techniques for automatic colorization, all of them being convolutional neural networks. The color space which is used during colorization is found to have an influence on the end result, and CIELab is chosen as the most promising one. Colorization can be seen as a mapping between a grayscale input image and an output color per pixel. While viewing this as a regression problem yields satisfactory results, it is found that implementing it as a classification problem, with the output of the network being a probability distribution over the colors in a discretized color space, yields much better results. The averaging problem, which causes networks to tend towards undersaturated or sepia tones, is shown to be solved using the classification approach. Using an annealed mean with a parameter setting for picking either the mean or mode of the probability distribution yields very convincing results. Finally, a model architecture using dilated convolutions in the upscale reconstruction pipeline of the network is found to generate the most convincing colorization.


\end{abstract}


% For peerreview papers, this IEEEtran command inserts a page break and
% creates the second title. It will be ignored for other modes.
\IEEEpeerreviewmaketitle


\section{Introduction}\label{sec:intro}

%Intro
%Reason for image colorization
%
%Types of image colorization (from literature review, but shorter?)
%by example
%by user input
%by a trained machine
%
%Research problem: 
%how to colorize an image using NN
%How to pick a color: mean in color space is sepia
%
%What is in the rest of this paper


\IEEEPARstart{I}{n} this paper a deep neural network approach to colorize grayscale images is introduced. Specifically, given a grayscale image, it can generate all the data needed to present a color version of the same image, without input of the user. 

There is a substantial amount of grayscale photographic material, from before the advent of the color camera. Color images may have a greater psychological impact on people than their grayscale counterparts.
In addition, automation of the colorization process can be applied in real time to grayscale video. Specifically (infra-red) surveillance cameras often save video in grayscale format. With the automated colorization techniques it may be possible to generate real-time color video, such that a human may more quickly understand what is seen in a video, in addition to a decrease in file size. In order to do this, it is required that the algorithm can colorize an image without intervention of a human.\\

As will be explained below, non-machine learning techniques are available for automatic image colorization. However, the drawback of these techniques is that they require either an image comparable to the grayscale image in terms of content, or user input on what color to use for different parts of the image. 
While this makes things much easier than the hand-made solution with photo editing software, it still requires a human to specify these inputs, which in turn disallows a real-time solution. This makes the case for a trained machine (i.e. neural network), which uses pre-supplied training data to learn, and while actually colorizing an image does not require additional input from a human.

Identified by the types of input required for colorization, there are three approaches to make the computer convert a grayscale image to a color image. Every approach uses the texture and intensity gradients present in the image to link a part of an image to either a learned or specified color. The three approaches are as follows:

\textbf{Colorize by example:} Image colorization can be performed by using a color image that is related to the grayscale image, in the sense that it contains similar objects with similar colors. The patterns in the color image are compared to those in the grayscale image, directly linking colors to the grayscale patterns and finding equal patterns in the to be colorized grayscale image. 
For example: to colorize a grayscale image of a zebra in a Savannah, another image of a zebra in a Savannah is required. The more comparable the image, the better the result is. This method is used in \cite{Charpiat}, \cite{Gupta} and \cite{Zheng}.

\textbf{Colorize by user input:} In this approach, the user specifies the color of different parts of the image by hand. While this is quite labor intensive, it guarantees that correct colors are used for the different segments in an image. 
As opposed to the \textit{colorize by example} method, this method cannot colorize images incorrectly purely based on similarity between contrast patterns. However, if the user does not know the original colors of the grayscale image, colorization is not possible. This method is used in \cite{Horiuchi} and \cite{Levin}.

\textbf{Colorize using a trained machine:} Techniques like convolutional neural networks allow training of a machine to recognize specific patterns in an image and coupling the recognized pattern to a color. This requires supplying the machine with training images of which both grayscale and color versions are available. 
After the training of the machine is completed, no human is needed to colorize an image since all information needed is contained within the system. This leads to a vast decrease in time consumption during colorization. One of the biggest downsides of this method is that objects are colorized based on experience with different objects of the same class. 
However, not all objects can be colorized purely based on their class of object, i.e. an object of class ``car" can have multiple colors, and during training multiple of these colors are shown to the machine, leading to an ambiguous mapping between object and color. The neural network method is used in \cite{Cheng}, \cite{Ho}, \cite{Krizhevsky} and \cite{Dahl}.\\

The research presented in this paper dives into how to best colorize an image using a Convolutional Neural Network (CNN). The question as to what is actually the best colorization of an image does not have one final answer. 
One answer could be that the best colorization would be the one that most people would be unable to distinguish from the original color image. This immediately involves the opinion of a human, warranting for example the use of a version of the Turing test, where for both ground truth color image and colorized image are shown to a human. The human in turn needs to pick which it thinks is colorized and which is the original. 
Although this makes quantification of the quality possible, it is labor intensive and a direct (mathematical) link between the way the system colorizes an image and result quality is not apparent. This makes it impossible to use for any gradient-based method of training the neural network. 

Another way is to mathematically define a quality measure, probably involving the ground truth color image and the colorized image, putting a number to the difference between these two. 
One of the most used error measures is the sum of the squared errors, calculating the difference in color per pixel, squaring every individual component and summing to obtain the final measure. 
This approach is natural to start with, but leads to one of the biggest unsolved problems encountered in automatic image colorization: the sepia and saturation problem.

The sepia and saturation problem arises from the following: a neural network is shown numerous examples of a certain object, of which it needs to deduce color from a grayscale image of this object. 
A trained neural network, in order to have the lowest possible cumulative error for all the training images, is encouraged to take some sort of mean color of all examples of the object in the training data set, which leads to the lowest mean squared error. For most objects that are linked to a certain color (such as bricks or a cloudless sky), this leads to undersaturated results, as illustrated in figure \ref{fig:colorspacesaturation}. 
For objects that are not inherently linked to a certain color, this leads to a mean that lies somewhere in the center of the color space, which in practice often leads to a generic sepia tone.\\

\begin{wrapfigure}{r}{0.37\textwidth}
	\vspace{-20pt}
	\begin{center}
		\includegraphics[width=0.37\textwidth]{saturation.png}
	\end{center}
	\caption{\label{fig:colorspacesaturation} The CIE 1931 color space containing all colors detectable by a human. When training a network to colorize to certain target colors of a green/blue object (black dots in image) minimizing the sum squared error, an average color will result (white dot in image), which most of the time leads to a less saturated color, i.e. more to the center of this color space. When the original colors are even more scattered, for example cars or clothes which do not inherently have a single color, a generic sepia color usually results.}
	\vspace{-10pt}
\end{wrapfigure}
	
In the rest of the paper the research is detailed on the problem of automatic image colorization using neural networks, starting with a literature review containing relevant other neural networks in section \ref{sec:litreview}. Multiple ideas are derived from these previous works. In section \ref{sec:method} these ideas in combination with new insights are described, detailing the method and different architectures to try. After that the results, a discussion of the results, a conclusion and recommendations are given.
	
	





\section{Literature Review}
Automatic image colorisation is a problem that has been attempted to solve using different approaches. At the basic level, there are three approaches to make the computer convert a grayscale image to a color image. Every approach uses the texture and intensity gradients to link to either a learned or specified color. The three approaches are:

image colorisation can be performed by using a target image that is related to the grayscale image in that it contains similar objects with similar colors. 
The approach the authors will use is of a trained neural network, tha


\begin{enumerate}
	\item hello
	
\end{enumerate}










\section{Method} \label{sec:method}
%Method
%Prerequisites (things used by all networks)
%Data set (fruit), (landscape)
%Input size (128X128)
%
%Color spaces + which one to use :
%RGB (luminosity not separated from color)
%HSV (circular domain)
%YCbCr (OK)
%CIELab (OK)
%first layer as input, second and third layer as output
%
%Architecture (not what it is but why WE use it)
%General discription (how to colorize an image with an NN)
%Features used by all networks
%ReLu, weight initialization, padding, kernel size
%Feature extraction
%Reconstruct
%Concatenate
%Dilated convolution
%Color generation
%Two feature maps
%blur
%Classification
%k-means
%annealed mean
%gaussian blur
%
%Loss function
%Squared error
%Class rebalancing (histogram dataset)
%Cross entropy
%Class rebalancing
%Architectures used:
%Dahl, Compact, Dahl_classifier, Dahl_zhang, Zhang
%
%Training method
%nesterov momentum
%adadelta

This section contains the various techniques used for the colorization problem.
\subsection{Dataset and Input}

As an input to the network a large amount of images are needed.
However, due to computational limitations, restrictions had to be made.
One of these restrictions resulted in a selection of images based on a certain category; fruit and landscape images.
The fruit dataset contain a rich amount of colors, straining the networks requirements.
This strain is implied by the ambiguity of fruit, being able to contain a vast set of different colors, difficult to predict purely based on a grayscale image.
The landscape dataset is more straight forward, containing less saturated colors and lacks fine details.

The datasets where generated using the popular image website {\color{red} Flickr (XXX)}.
Using their freely available API a program was made that retrieved images in the required resolution and kept track of images retrieved, to avoid duplicates.
The images are then collected in batches and stored in Numpy arrays as input for the network.

The datasets retrieved had to be checked on incorrect images. To solve this problem a web application was made that enabled us to check images on defects. A detailed description can be found in {\color{red}appendix (XXX)}

This resulted in 2 datasets, which are summarized in table \ref{tab:dataset}.

\begin{table}[h!]
	\centering
	\caption{Datasets used for training and validation of the various networks}
	\label{tab:dataset}
	\begin{tabular}{|l|l|l|}
		\hline
		Dataset   & Training Images & Validation images \\ \hline
		Fruit     & 6000            & 1000              \\ \hline
		Landscape & 34000           & 5000              \\ \hline
	\end{tabular}
\end{table}

The network input are 128x128 pixel grayscale images, which are propagated through the network in a given batch sizes. Using batches instead of one image at a time reduces computation time because better use is made of the parallelization of modern computer architectures, but more importantly the gradients calculated using a batch of images are a better estimate of the gradient for the entire training set, thus leading to a more stable gradient descent\cite{ioffe2015batch}. Making the batch size too large makes the search to the global minimum less stochastic, thus making it converge to a local optimum more quickly.

The network is trained to be able to colorize an image, but the way it outputs the colorized image is dependent on the color space used during training. There are several options available:\\

\textbf{RGB} This widely used color space specifies an intensity for the channels red, blue and green. The greatest drawback of using RGB is that the color is not separated from the luminosity. In this way, the network needs not only to output a hue and saturation of a color, but also the luminosity itself. This makes it a much tougher challenge to output an image that resembles a colorized version of the grayscale image. A visualisation of the RGB space is given in figure \ref{fig:RGB}

\textbf{HSV} Specifying the hue, saturation and value, uncouples the luminosity (value) from the color (hue and saturation). Furthermore decoupling the saturation could allow specifically tackhe ling the sepia and saturation problem as described in section \ref{sec:intro}. However, the color space is periodic in the hue axis. From the numerical perspective, this leads to an ambiguous error specification, since for one difference between target and network output color, two directions of improvement are equally valid, thus rendering gradient descent impossible.

\textbf{YCbCr} The Y channel contains the luminosity of the image, while Cb and Cr layers are the chroma blue and chroma red layers respectively. While providing a separate luminosity channel, it was found that for different Y values, a given Cb and Cr combination does not specify the same color. This makes the error in the color specification dependent on the luminosity of the image. A visualization of the color space can be found in \ref{fig:YCbCr}.

\textbf{CIELab} Similarly to the YCbCr color space, the L channel specifies the luminosity, the a and b layers the color. In contrast with the YCbCr space, a given a and b value specify a color, while the luminosity only determines how light the color is. However, for a single L layer not all a and b value translate to colors that can be represented in the RGB color space (as displayed by computer monitors). The CIELab color space is visualized in figure \ref{fig:CIELab}\\

To summarize, the HSV, YCbCr and CIELab color spaces all have the possibility of using one channel as input to the neural network, while requiring only two outputs of the network that instead of three with the RGB color space. Combined with the input, these two outputs allow reconstructing a color image. According to {\color{red}fdsafkldsjkflsa}, the CIELab color space has been found to give the most perceptually correct results for a colorization network {\color{red} Cite die ene paper van dawud waar ze dit zeggen en zeg wie het was!!}\\

\begin{figure}
	\centering
	\begin{subfigure}[b]{0.32\textwidth}
		\includegraphics[width=\textwidth,trim={125px 75px 125px 75px},clip]{RGB}
		\caption{The RGB colorspace}
		\label{fig:RGB}
	\end{subfigure}
	~ %add desired spacing between images, e. g. ~, \quad, \qquad, \hfill etc. 
	%(or a blank line to force the subfigure onto a new line)
	\begin{subfigure}[b]{0.32\textwidth}
		\includegraphics[width=\textwidth,trim={125px 75px 125px 75px},clip]{YCbCr}
		\caption{The RGB colorspace represented in the YCbCr colorspace}
		\label{fig:YCbCr}
	\end{subfigure}
	~ %add desired spacing between images, e. g. ~, \quad, \qquad, \hfill etc. 
	%(or a blank line to force the subfigure onto a new line)
	\begin{subfigure}[b]{0.32\textwidth}
		\includegraphics[width=\textwidth,trim={175px 75px 150px 75px},clip]{CIELab}
		\caption{The RGB colorspace represented in the CIELab colorspace}
		\label{fig:CIELab}
	\end{subfigure}
	\caption{The different colorspaces used in the different networks}\label{fig:animals}
\end{figure}


\subsubsection{Neural network properties}
\label{sec:nnproperties}
Neural networks contain a vast amount of different available parameters to tweak the networks behaviour. Throughout the report some of these parameters are kept constant for each network architecture. 

When using a network without pre-trained weights, the networks weights are initialized. This is done using a Glorot Uniform distribution \cite{Glorot} as initialization of the weights. This weight initialization method samples from a uniform distribution with its variance scaled depending on the ingoing and outgoing data. %?

The network consists mainly of convolutional layers. All convolutional layers, except some output layers, use ReLu non-linearities. 

Kernel sizes in the convolutional layers of the network are 3x3 kernels. This kernel size is based upon VGG16's \cite{Simonyan} kernel size.

Stride in the convolutional layers is 1 by default. However, in the classification network {\color{red} ref(XXX)} stride is used as a way to downsample the resolution of the image, comparable to the function of max pooling. %?

Padding is required when using convolutional layers due to the fact that border information of the image is lost when convolving. This property is set to keep the output resolution the same as the input resolution. 


\subsection{Feature Extraction}
To be able to recognize certain objects in grayscale images, object dependent features are extracted. 
This is done in the first part of the convolutional neural network, up until the bottleneck of the network.
To extract these features convolutional layers are used with a varying amount of feature maps. A part of this feature extraction is reducing local features to global features, making them spatially invariant. To accomplish this, Max-pooling is widely used. Another technique available is using an appropriate amount of stride when convolving. {\color{red}reference??}.

This results in spatially invariant feature maps in the bottleneck, only indicating what is in the image, but has no spatial information. To retrieve this spatial information, proper reconstruction of the image has to be done. This is expanded upon in section \ref{sec:reconstruction}.

\subsection{reconstruction}
Before the bottleneck, the image is reduced to a set of features, containing no spatial information. To be able to colorize the image, spatial locations of the object are mandatory. For reconstruction, various methods are used to retrieve the original image resolution.

First of all, the feature maps can be upscaled using linear upscaling. However, to retrieve the original image not only the features but also the information of where those features are coming from have to be retrieved. This is done by concatenating the layers of the same resolution before the bottleneck with the upscaled features after the bottleneck. Then an convolutional layer is used to merge these features in feature maps of the upscaled resolution. This process is repeated until the original image resolution is retrieved.

{\color{red}Another technique used is the use of dilated convolutions.} %Jopie.}

\subsection{Color Generation}
In the final layers of the network, the original image color layers have to be reconstructed. Two different methods are used throughout the paper; Construction using two feature maps and classification.

The construction using two feature maps is a direct result of retrieving the original image resolution through the reconstruction of the image. As a final layer, a convolutional layer is used that maps to two feature maps, which represent the two color layers that are finally used to create a colorized image.

As an aid to the colorization process, a Gaussian Blur is used on the color layers of the original image. Colorization does not require precise pixel by pixel colorization, because colors in images mostly appear in sets of pixels. When blurring, this enables the network to more easily converge towards a solution by reducing the noise in the color of the pixel set. Note that blurring the color layers does not reduce the image fidelity. This is due to the fact that the luminosity layer contains the details and contrast of the image, which is subsequently merged with the output color layers of the network.

 





 
\subsection{Model Architectures}
A major part in creating a successful neural network is finding a suitable network architecture. For image classification convolutional neural networks are widely used with success \cite{Krizhevsky,Szegedy,Simonyan}. For our purpose a main feature of the convolutional network is that the image has to be reconstructed again, to retrieve spatial information.

{\color{red} 
In total a set of three convolutional network architectures are used. An architecture based upon Dahl \cite{Dahl}, a pre-trained VGG16 \cite{Simonyan} architecture {\color{red} this is actually Dahl,  not sure about the actual amount of layers used for the reconstruction} and a classification architecture \cite{Zhang}. The architectures of the network are split up in a feature extraction part and a reconstruction part, which will be expanded upon in the following sections.}\\ 
\\%weet neit of je dit mag zeggen..

{\color{red}
\textbf{Feature extraction}

%%Check op diepgang.

\subsubsection{Dahl}%weet geen naam}


This convolutional network is based upon the architecture used by Dahl \cite{Dahl}. It contains several convolutional layers, which use a 3x3 kernel throughout the network. After a set of convolutions a batch normalization is done followed by a max pool layer. Batch normalization is added such that in the reconstruction of the image the concatenated layers are in the same order of magnitude. The architecture was modified to fit the input dimensions. The network is an untrained network, having Glorit uniform distributed \cite{Glorot} initialized weights, meant to be trained simultaneously with the rest of the network. At the bottleneck of the architecture, the resolution of the feature maps are reduced to 16x16 pixels.
 
%Figure toevoegen van het netwerk

\subsubsection{VGG16}
A substantial amount of pre-trained networks are available, trained on the ImageNet classification database. The architecture used is based upon VGG16 \cite{Simonyan}, which uses a 3x3 kernel throughout the network. This network has a proven architecture, and can be obtained with pre-trained weights. Modifications on the network where required to fit the input dimensions. VGG16 is used in classification of RGB images, while our network only needs one input, a grayscale image. The pre-trained weights of the three input maps where averaged to accept a single grayscale input image. The VGG16 architecture features several convolutional layers followed by max pooling. Batch normalization was added before every max pool layer, such that in the later concatenation of the layers the values are in the same order of magnitude. At the bottleneck of the architecture the feature maps have a resolution of 16x16 pixels. 

\subsubsection{Classification}
The classification architecture shares much similarities with the VGG16 architecture. The main difference can be found in the final layers of the network. This is due to the fact that classification is used rather than direct reconstruction of the wanted color layers. This classification is subsequently used to colorize the image. \\
\\
\textbf{Reconstruction}

For reconstruction of the image, linear up-scaling is used. The reconstruction begins after the bottleneck of the convolutional network is reached, where the resolution of the feature maps is 16x16 pixels. The up-scaled information is concatenated with the convolutional layer before the bottleneck that matches the up-scaled layer resolution. Then a convolutional layer is used for feature extraction of the concatenated layer. A batch normalization is applied and the processes is repeated until the original image resolution is retrieved. For both the VGG16 and Dahl based architecture a final output layer is used with a 2 feature map output, which match to the corresponding colour output layers.

For the classification the image is reconstructed to its original resolution. However, first a convolutional layer using a 1x1 kernel is used that has the same amount of feature maps as the required number of possible color classification bins. Then, these feature maps, representing discretized colours, %jopie jou expertise.


A detailed representation with of the various architectures is given in figure {\color{red}(XXX)}}

\subsection{Loss function}

\begin{wrapfigure}{R}{0.5\textwidth}
	\vspace{-20pt}
	\begin{center}
		\includegraphics[width=0.48\textwidth]{hist}
	\end{center}
	\caption{\color{red} The histogram of the total fruit dataset}
	\vspace{-10pt}
\end{wrapfigure}

\subsection{Training method}
For convolutional neural networks stochastic gradient decent {\color{red}(SGD)}, sometimes together with momentum, is commonly used for updating the weights and biases \cite{IizukaSIGGRAPH2016} \cite{Simonyan}. The used hyper-parameters, especially the learning rate, requires careful tuning when using SGD. Often scheduled learning rate, that is monotonically decreasing depending on the epoch results in the best results. 




\section{Results}\label{sec:results}
%Results
%Show effect of colorspaces, blur, annealed mean
%Compare results of architecture (images, error plot, feature maps)

In section \ref{sec:method} all the different techniques and methods used resulted in multiple experiments. This sections shows the different results of each experiment, setting the foundation for the discussion in section \ref{sec:discussion}.

\subsection{Colorspace}
To asses the effect of different colorspaces, a comparison is made between CIELab and YCbCr. The compact network is trained on the landscape set for both the YCbCr and CIELab color space, for a total of 20 epochs. The result is shown in figure \ref{fig:colorspacecomparison}.

\subsubsection{Gaussian blur}
The comparison between different Gaussian blur kernel standard deviations is seen in figure \ref{fig:blur}. The comparison was done with the compact network, trained on the landscape dataset. Note that the standard deviation only comes into play during training, since the target output layer is blurred the target less noisy. It is clear that the case with $\sigma=0$, i.e. no blur, the network was unable to find any colors. Using a blur radius of $\sigma=3$ or $\sigma=5$ leads to much better results. Between these two, the $\sigma=3$ case was chosen to be the best, since it seems more inclined to pick more saturated colors. A too high blur radius will also make the training less effective since the link between texture and color becomes weaker at the edges of objects.

\begin{figure}[h]
	\centering
	\includegraphics[width=0.6\textwidth]{blur}
	\caption{Blur}
	\label{fig:blur}
\end{figure}

\begin{figure}[h]
	\centering
	\includegraphics[width=0.6\textwidth]{YCbCr_vs_CIELab}
	\caption{YCbCr vs CIELab}
	\label{fig:YCbCr_vs_CIELab}
\end{figure}

In section \ref{sec:method} all the different techniques and methods used resulted in a selection of five final network architectures with distinct properties. In this section the results of these networks are shown in the form of colorization attempts by the various neural networks.

\begin{figure}[h]
	\centering
	\includegraphics[width=0.9\textwidth]{set2}
	\caption{Results}
	\label{fig:results}
\end{figure}



\section{Discussion}

Defining a quantative performance measure of the networks is outside of the scope of this paper. Therefore a comparison of the results of the five architectures is mainly done by sight.

%comparison of datasets fruit and landscape, on compact net . 
\subsection{Comparing dataset}
The compact network was


%discussion on difference between YcbYcr and Cielab Effect of blur
\subsection{Comparing colorspace and effect of blur}

%overfitting of vgg16 so dataset too small
% using compact architectures better results
\subsection{Comparing network complexity}

%comparison between dilation, concat and dilation+concat
\subsection{Comparing feature localization}

%final comparison between all architectures, show massive image.
\subsection{Final comparison}

%conclusion on best network







%Gaussian blur
%It is clear that the case with $\sigma=0$, i.e. no blur, the network was unable to find any colors. Using a blur radius of $\sigma=3$ or $\sigma=5$ leads to much better results. Between these two, the $\sigma=3$ case was chosen to be the best, since it seems more inclined to pick more saturated colors. A too high blur radius will also make the training less effective since the link between texture and color becomes weaker at the edges of objects.
\section{Conclusion}
%Conclusion
%Recommendations: What's next
%hyperparameters research (momentum etc)
%longer training
%bigger training set
%extend usage to all images

Various techniques exist to colorize grayscale images, however, each requiring a significant amount of human intervention. A promising solution is through the use of neural networks. Throughout this report various methods and properties of neural networks are compared and discussed. It is shown that the many different combinations in components and properties of neural networks can give a vast difference in results, proving the sensitivity of the neural network to architecture changes. It is also clear that no single architecture is best, since different architectures give different results for different images. In the end, the compact classifier dilated is picked as the best architecture. Furthermore, it is concluded that using the CIELab color space yields slightly more visually pleasing results compared to using the YCbCr color space. 


\section{Recommendations}

%Recommendations: What's next
%hyperparameters research (momentum etc)
%longer training
%bigger training set
%extend usage to all images

It can be concluded that using CNNs for automatic colorization offers promising results. However, a lot of improvement can still be made on the current approach. As mentioned before overfitting is evident on the more complex datasets. Increasing the dataset or even training on large datasets such as IMAGENET \cite{deng2009imagenet} would give an insight on how the colorization algorithms generalises over multi-class training sets.\\

Another addition to the research mentioned in this paper is that of model averaging. Model averaging combines the outputs of multiple CNNs with different architectures, averaging the output probabilities, resulting in a more accurate final output. Figure \ref{fig:moreresults} and \ref{fig:results} show that it is difficult to point out one best architecture. Different architectures score high on different images. Therefore model averaging would make the best of all architectures and average them, resulting in a good overall result.\\

It would also be interesting to see how one of the CNNs mentioned above, performs in combination with the 'colorization by user input' technique mentioned in section \ref{sec:intro}. An application could be to colorize a large image, while the output of the colorization neural network for a small version of the image could be used as the color input for colorization of the large version of the image.\\

The color generation by classification does not result in one single color, but rather in an estimate of the probability distribution over all possible colors. This probability distribution is given for one single pixel, but it could be interesting to look at the joint probability distribution given for a region of pixels, to ensure a smooth distribution of color over one same-color part of an object in the image.



\appendices
\begin{table}[]
	\centering
	\caption{Parameters of the compared architectures.\label{tab:architectures}}
	\begin{tabular}{|l|l|l|l|l|l|}
		\hline
		& \textbf{compact}                                                  & \textbf{compact classifier}                                       & \textbf{compact classifier dilated}                               & \textbf{\begin{tabular}[c]{@{}l@{}}compact classifier\\  dilated + concat\end{tabular}} & \textbf{\begin{tabular}[c]{@{}l@{}}vgg16 compact \\ + classifier\end{tabular}} \\ \hline
		\textbf{dataset parameters}        &                                                                   &                                                                   &                                                                   &                                                                                          &                                                                                \\ \hline
		\textit{Color space}               &                                                                   &                                                                   &                                                                   &                                                                                          &                                                                                \\ \hline
		\textit{Fruit dataset}             & Yes                                                               & Yes                                                               & Yes                                                               & Yes                                                                                      & Yes                                                                            \\ \hline
		\textit{Landscape dataset}         & Yes                                                               & No                                                                & No                                                                & No                                                                                       & No                                                                             \\ \hline
		\textbf{Hyperparameter}            &                                                                   &                                                                   &                                                                   &                                                                                          &                                                                                \\ \hline
		\textit{Training method}           & \begin{tabular}[c]{@{}l@{}}ADADELTA\\  with momentum\end{tabular} & \begin{tabular}[c]{@{}l@{}}ADADELTA\\  with momentum\end{tabular} & \begin{tabular}[c]{@{}l@{}}ADADELTA\\  with momentum\end{tabular} & \begin{tabular}[c]{@{}l@{}}ADADELTA with \\ momentum\end{tabular}                        & \begin{tabular}[c]{@{}l@{}}ADADELTA\\  with momentum\end{tabular}              \\ \hline
		\textit{Epochs trained}            &                                                                   &                                                                   &                                                                   &                                                                                          &                                                                                \\ \hline
		\textit{Batch size}                &                                                                   &                                                                   &                                                                   &                                                                                          &                                                                                \\ \hline
		\textbf{Architecture properties}   &                                                                   &                                                                   &                                                                   &                                                                                          &                                                                                \\ \hline
		\textit{Front end module}          & trained from scratch                                              & trained from scratch                                              & trained from scratch                                              & trained from scratch                                                                     & VGG16 front end                                                                \\ \hline
		\textit{Max pool}                  & Yes                                                               & Yes                                                               & No                                                                & No                                                                                       & Yes                                                                            \\ \hline
		\textit{Strides}                   & No                                                                & No                                                                & Yes                                                               & Yes                                                                                      & No                                                                             \\ \hline
		\textit{dilation}                  & No                                                                & No                                                                & Yes                                                               & Yes                                                                                      & No                                                                             \\ \hline
		\textit{Concatenation}             & Yes                                                               & Yes                                                               & No                                                                & Yes                                                                                      & Yes                                                                            \\ \hline
		\textit{Kernel size}               & 3x3                                                               & 3x3                                                               & 3x3                                                               & 3x3                                                                                      & 3x3                                                                            \\ \hline
		\textit{Activation function}       & ReLu                                                              & ReLu                                                              & ReLu                                                              & ReLu                                                                                     & ReLu                                                                           \\ \hline
		\textit{Batch norm}                & Yes                                                               & Yes                                                               & Yes                                                               & Yes                                                                                      & Yes                                                                            \\ \hline
		\textbf{Classification properties} &                                                                   &                                                                   &                                                                   &                                                                                          &                                                                                \\ \hline
		\textit{Annealed mean temperature} &                                                                   &                                                                   &                                                                   &                                                                                          &                                                                                \\ \hline
		\textit{K-nearest neighbour}       & -                                                                 & 10                                                                & 10                                                                & 10                                                                                       & 10                                                                             \\ \hline
		\textit{K-nearest neighbour sigma} & -                                                                 & 5                                                                 & 5                                                                 & 5                                                                                        & 5                                                                              \\ \hline
	\end{tabular}
\end{table}



\ifCLASSOPTIONcaptionsoff
  \newpage
\fi



\FloatBarrier
\bibliography{./Bibliography/Bibliography}
\bibliographystyle{ieeetr}

 \end{document}


