\section{Introduction}\label{sec:into}

%Intro
%Reason for image colorization
%
%Types of image colorization (from literature review, but shorter?)
%by example
%by user input
%by a trained machine
%
%Research problem: 
%how to colorize an image using NN
%How to pick a color: mean in color space is sepia
%
%What is in the rest of this paper


\IEEEPARstart{I}{n} this paper a deep neural network approach to colorize grayscale images is introduced. Specifically, given a grayscale image, it can generate all the data needed to present a color version of the same image, without input of the user. 

There is a substantial amount of grayscale photographic material, from before the advent of the color camera. Color images may have a greater psychological impact on people than their grayscale counterparts, which gives rise to the need for automatic colorization algorithms. 
In addition, automation of the colorization process can be applied in real time to grayscale video. Specifically (infra-red) surveillance cameras often save video in grayscale format. With the automated colorization techniques it may be possible to generate real-time color video, such that a human may more quickly understand what is seen in a video, in addition to a decrease in file size. In order to do this, it is required that the algorithm can colorize an image without intervention of a human.

As will be explained below, non-machine learning techniques are available to automatic image colorization. However, the drawback of these techniques is that they require either an image comparable to the grayscale image in terms of content, or user input on what color to use for different parts of the image. While this makes things much easier than the hand-made solution with photo editing software, it still requires a human to specify these inputs, which in turn disallows a real-time solution. This makes the case for a trained machine (i.e. neural network), which uses pre-supplied training data to learn, and while actually colorizing an image does not require additional input from a human.

Identified by the types of input required for colorization, there are three approaches to make the computer convert a grayscale image to a color image. Every approach uses the texture and intensity gradients present in the image to link a part of an image to either a learned or specified color. The three approaches are as follows:

\textbf{Colorize by example:} Image colorization can be performed by using a similar color image that is related to the grayscale image in that it contains similar objects with similar colors. The patterns in the color image are compared to those in the grayscale image, directly linking these colors to the patterns. For example: to colorize a grayscale image of a zebra in a Savannah, another image of a zebra in a Savannah is required. The more similar the image, the better the result is. This method is used in \cite{Charpiat}, \cite{Gupta} and \cite{Zheng}.

\textbf{Colorize by user input:} In this approach, the user specifies the color of different parts of the image by hand. While this is quite labor intensive, it guarantees that correct colors are used for the different segments in an image. As opposed to the \textit{colorize by example} method, this method cannot colorize images incorrectly purely based on similar contrast patterns. However, if the user does not know the original colors of the grayscale image, colorization is not possible. This method is used in \cite{Horiuchi} and \cite{Levin}.

\textbf{Colorize using a trained machine:} Techniques like convolutional neural networks allow training of a machine to recognize specific patterns in an image and coupling the recognized pattern to a color. This requires supplying the machine with training images of which both grayscale and color versions are available. After the training of the machine is completed, no human is needed to colorize an image since all information needed is contained within the system. This leads to a vast decrease in time consumption during colorization. One of the biggest downsides of this method is that images are colorized based on experience with different objects of the same class as in the grayscale image. However, not all objects can be colorized purely based on their class of object, i.e. an object of class ``car" can have multiple colors, and during training multiple of these colors are shown to the machine, leading to an ambiguous mapping between object and color. This method is used in \cite{Cheng}, \cite{Ho}, \cite{Krizhevsky} and \cite{Dahl}.


It was found that if during training enough similar objects with different colors are given to the neural network, a generic sepia tone results. 



Reason for image colorization
	
	
	
Types of image colorization (from literature review, but shorter?)
	by example
	by user input
	by a trained machine
	
	
	
Research problem: 
	how to colorize an image using NN
	How to pick a color: mean in color space is sepia
	
	
	
What is in the rest of this paper
	
	
	
	




