\section{Introduction}
The main problem that is going to be addressed is that of automatic image colorization. There is a substantial amount of black and white photographic material, such as old pictures and movies or videos of (political) speeches. Currently, these images are colorized, if at all, by hand using photo editing software, which is very labor intensive. However, color images may have a greater impact on people viewing these than their grayscale versions, which gives the need for automatic colorization algorithms.

In addition, automation of this process can be applied to real time colorization of video. Specifically (infra-red) surveillance cameras often save video in grayscale format, to save in file size or because the low-light conditions don't allow a good color representation. With the automated colorization techniques it may be possible to generate color video on-demand while viewing, such that a human may more quickly understand what is seen in a video. In order to do this, it is required that the algorithm can colorize an image without intervention of a human, as a human would make real-time solutions impossible.

As will be explained in Section \ref{sec:litreview}, non-machine learning techniques are available to automatic image colorization. However, the drawback of these techniques is that they require either an image comparable to the grayscale image in terms of content, or user input on what color to use for different parts of the image. While this makes things much easier than the hand-made solution with photo editing software, it still requires a human to specify these inputs, which in turn disallows a real-time solution. This makes the case for a trained machine (i.e. neural network), which uses pre-supplied training data to learn, and while actually colorizing an image does not require additional input from a human. A large training data set can easily be generated from color images, such that gathering training data is not an issue.
