\section{Literature Review} \label{sec:litreview}
Automatic image colorization is a problem that has been attempted to solve using different approaches. Separated by input type, there are three approaches to make the computer convert a grayscale image to a color image. Every approach uses the texture and intensity gradients of the image to link a part of an image to either a learned or specified color. The three approaches are:

\paragraph{Colorize by example:} Image colorization can be performed by using a target image that is related to the grayscale image in that it contains similar objects with similar colors. The objects in the grayscale image are compared to the target image via the patterns of the similar objects to find these what color the similar patterns should have. For example: to colorize a grayscale image of a zebra in the Savannah, another image of a zebra in a Savannah is required to create a good colorization. The more similar the image, the better the result is. This method is used in \cite{Charpiat}, \cite{Gupta} and \cite{Zheng}.
	
\paragraph{Colorize by user input:} In this approach, the user specifies the color of different parts of the image by hand. While this is quire labor intensive, it guarantees that correct colors are used for the different objects in an image. For example, in the previous method the color of grass may be specified as green in the target image, whereas in the grayscale image the grass is actually colored brown. Due to the similarities in contrast the grass will thus be colored green, but in the colorize by user input this will not be a problem. However, if the user does not know the original colors of the grayscale image, colorization is not possible. This method is used in \cite{Horiuchi} and \cite{Levin}.
	
\paragraph{Colorize using a trained machine:} Techniques like convolution neural networks allow training of a machine that recognizes specific patterns in an image and couples the recognized pattern to a color. This requires the use of training images of which both grayscale and color versions are available. Since no human intervention is required after the training is completed, this may lead to a vast decrease in time consumption during colorization. However, not all objects have the same color always, i.e. the same car model can have different colors, and a trained machine cannot identify what color a specific object is based on its class. It was found that a generic sepia tone results if enough examples of an object are in the training set. This method is used in \cite{Cheng}, \cite{Ho}, \cite{Krizhevsky} and \cite{Dahl}.

This specific image task requires a deep neural network to be able to couple color to features in black and white images, most likely in the form of a convolutional neural network, as used in the literature. More literature about convolutional neural networks can be found in \cite{GoodfellowBOOK}. To solve the sepia color problem it is proposed to use generative adversarial networks, to influence the discriminator to be able to generate the right colors where there could be multiple solutions. generative adversarial networks is described by \cite{Goodfellow} and \cite{Radford}. Another method to tackle this problem are variational Auto-encoders, altering a direct copy of the output, (variational) auto encoders are described by \cite{Gregor}, \cite{Kingma} and \cite{GoodfellowBOOK}.










